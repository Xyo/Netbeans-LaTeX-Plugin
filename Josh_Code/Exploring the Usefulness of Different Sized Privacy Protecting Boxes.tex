\documentclass{article}
\begin{document}

\title{Exploring the Usefulness of Different Sized Privacy Protecting Boxes}
\author{  }
\date{ }

\begin{abstract}
Often when given data that needs to remain private such as medical information, exact values in the data space are replaced with appropriate boxes. These boxes can be used to calculate a range for particular statistic, while still protecting the privacy of individuals.

There are two main factors that we want to take into account when forming these boxes: $ k $--anonymity and $ l $--diversity. Both of these characteristics will undoubtedly change depending on the sizes of the boxes. In this paper, we want to compare the usefulness of boxes being equal in size with that of different sized boxes based on the data.
\end{abstract}

\section{Introduction}

\paragraph{Preserving Privacy.} The idea of breaking data into boxes of any size:

Show notation for boxes of any size vs. boxes of the same size. 

\paragraph{Explain $ k $--anonymity}

\paragraph{Explain $ l $--diversity}

\paragraph{ }

\section{Subdivision into Equal Sized Boxes}

\section{Subdivision into Different Sized Boxes}


\end{document}
